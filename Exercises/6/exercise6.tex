\documentclass[11pt,a4paper]{article}

\usepackage[utf8]{inputenc}
\usepackage[english]{babel}
\usepackage[T1]{fontenc}

\usepackage{amsmath,amssymb,amsfonts}

\title{Computational Geometry - Voronoi Diagrams}
\author{Philip Munksgaard \\ Sebastian Paaske Tørholm \\ Ejnar Håkonsen}

\begin{document}
\maketitle

\section{Exercise 7.1}

We can create such a set by placing one of the points $p_0$ as the
center of a circle, and placing the remaining points such that they
all have the same distance to $p_0$ and are evenly spaced around
$p_0$, in effect forming a circle around $p_0$. Since all the points
around $p_0$ have the same distance to $p_0$ and to their neighbors,
they will all share an edge with the cell containing $p_0$ in the
Voronoi diagram. As such, that cell will have exactly $n-1$ vertices.

\section{Exercise 7.3}

\section{Exercise 7.10}

\section{Exercise 7.12}

\end{document}

