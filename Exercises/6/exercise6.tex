\documentclass[11pt,a4paper]{article}

\usepackage[utf8]{inputenc}
\usepackage[english]{babel}
\usepackage[T1]{fontenc}

\usepackage{amsmath,amssymb,amsfonts}

\title{Computational Geometry - Voronoi Diagrams}
\author{Philip Munksgaard \\ Sebastian Paaske Tørholm \\ Ejnar Håkonsen}

\begin{document}
\maketitle

\section{Exercise 7.1}

We can create such a set by placing one of the points $p_0$ as the
center of a circle, and placing the remaining points such that they
all have the same distance to $p_0$ and are evenly spaced around
$p_0$, in effect forming a circle around $p_0$. Since all the points
around $p_0$ have the same distance to $p_0$ and to their neighbors,
they will all share an edge with the cell containing $p_0$ in the
Voronoi diagram. As such, that cell will have exactly $n-1$ vertices.

\section{Exercise 7.3}
Assume we have $n$ numbers $x_1, \ldots, x_n$, and that we wish to sort these.

In $O(n)$ we find the minimal and maximal numbers, $x_{min}$
and $x_{max}$.

If we now normalize these points from $[x_{min} : x_{max}]$ to $[0 : \Pi]$, calling
$x_i$ normalized $x_i'$, we can now construct a 2-dimensional point-set.

Let $p_0 = (0,0)$, and let $p_i = (\cos(x_i'), \sin(x_i'))$. Each $x_i$ will
now correspond to a point on the unit circle. We now find a Voronoi diagram of
this point set, and wish to use this to find a sorted order for $x_1, \ldots,
x_n$.

If we look at $p_0$, it is equidistant to every other point, and thus its cell
will have an edge to each of them. Since the incident edges of a cell are
stored in cyclic order, it takes at $O(n)$ to transform this edge list to the
$x$'es in sorted order.

\section{Exercise 7.10}

We wish to find the two points of $P$ that are closest together.

We can determine the Voronoi diagram in $O(n \log n)$ time. The
Voronoi diagram contains at most $3n-6$ edges. Each edge connects two
faces, each containing one point, and by simply checking all such
edges to find the minimum distance between the two points they
connect, we can find the two points that are closest together. We know
that the faces containing the two points that are closest together
must share an edge, because otherwise there would be another point
somewhere in between, that is closer to the two points.

\section{Exercise 7.12}

%% Let the Voronoi diagram of a point set P be stored in a
%% doubly-connected edge list inside a bounding box. Give an algorithm
%% to compute all points of P that lie on the boundary of the convex
%% hull of P in time linear in the output size. Assume that your
%% algorithm receives as its input a pointer to the record of some
%% half-edge whose origin lies on the bounding box.

We start by adding the input point to the convex hull. We then want to
find the next face containing a point on the convex hull. First, we'll
use linear time to find an half-edge in the current face that's on the
bounding box (the amount of edges in a Voronoi diagram is linear in
the amount of nodes). When we've found an edge $e_b$ on the bounding
box, we simply take the next edge $e_1$ and find it's twin half-edge
$e_1'$ and corresponding face. The node contained in that face must be
in the convex hull as well. Then, $e_1'.next$ must be the half-edge
$e_b'$ that is on the bounding box of that face, and we can simply
repeat the procedure from above until we're back to $e_b$.

There is one special case that we have to take into consideration:
When a face in the Voronoi diagram is in a corner. First, we note that
there can be at most four of these cases. If, for some half-edge on
the bounding box, $e$, $e.next$ is also on the bounding box (ie. it
has no twin half-edge), we are in a corner and we simply use
$e.next.next$ instead. If the edge between two faces in the Voronoi
diagram fall exactly in a corner, we don't have to do anything.

\end{document}

