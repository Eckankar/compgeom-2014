\documentclass[11pt,a4paper]{article}

\usepackage[utf8]{inputenc}
\usepackage[english]{babel}
\usepackage[T1]{fontenc}

\usepackage{amsmath,amssymb,amsfonts}

\title{Computational Geometry - Exercises for week 2}
\author{Philip Munksgaard\\Sebastian Paaske Tørholm \\ Ejnar Håkonsen}

\begin{document}
\maketitle

\section{Exercise 3.8}

Adding diagonals to a DCEL can take $O(n)$ time since we have to
update the incident faces of all the edges in one half of the
diagonalized polygon, of which there may be up to $n-2$. Apart from
incident phases, adding a diagonal only requires us to modify two
\texttt{next} and \texttt{previous} pointers, as well as creating a
new face, all of which takes $O(1)$ time.

However, \texttt{MakeMonotone} changes the incident faces for each
edge at most once. Every time a diagonal is added, all the edges above
the diagonal are disregarded from the rest of the computations, and no
more diagonals are added to that part of the polygon. Since we know
that we add at most $n-2$ diagonals, the amortized cost of adding a
diagonal is thus $O(1)$.

\section{Exercise 3.14}

Given a simple polygon $\mathcal{P}$ with $n$ vertices and a point $p$
inside it, we wish to compute the region inside $\mathcal{P}$ that is
visible from $p$.

We start by computing the triangulation $\mathcal{T}$ of $\mathcal{P}$
as a DCEL and finding the triangle $t$ inside $\mathcal{T}$ that contains
$p$.

Since $p$ is inside $t$, we know that all corners, and therefore all
edges of $t$ are visible from $p$. We now recursively examine
neighbors of triangles that are known to be entirely visible, checking
to see if they are also visible.

Let $t_1$ consisting of vertices $v_1, v_2, v_3$ be a neighbor to an
entirely visible triangle $t_0$ consisting of vertices $v_0, v_1,
v_2$. We know that vertices $v_1 and v_2$ are visible from $p$, so we
just have to check whether $v_3$ is visible. This can be done simply
by checking whether the line segment $\{p, v_3\}$ intersects with the
line segment $\{v_1, v_2\}$. If that is the case, $v_3$ is also
visible from $p$ and $t_1$ is entirely visible from $p$ and we can
continue recursively. If not, there is some part $t_1$ that is not
visible from $p$. We thus have to split $t_1$ up into two triangles,
such that one of them is entirely visible from $p$ and the other is
entirely invisible. Doing so is a simple matter of producing the lines
from $p$ to $v_1$ and $p$ to $v_2$ and see if they intersect with the
line segments $\{v_2, v_3\}$ and $\{v_1, v_3\}$, respectively. If the
extended line from $p$ to $v_1$ intersects with $\{v_2, v_3\}$ at some
point $v_4$ we simply create a new entirely visible triangle $\{v_1,
v_2, v_4\}$ its neighbors (excluding the invisible part of
$t$). Analogously for $\{v_1, v_3\}$.

As each step in the recursive neighbor-checking takes $O(1)$ time, the
entire recursion only takes $O(n)$ time. Since triangulation takes
$O(n)$ time (using Chazelles algorithm) and finding the triangle that
contains $p$ takes $O(n)$ time (simply check each triangle to see if
it contains $p$, the entire thing takes $O(n)$ time. Finally, since
the only data structure we use is the DCEL, we only take up $O(n)$
space as well.

\end{document}

