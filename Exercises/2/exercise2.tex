\documentclass[11pt,a4paper]{article}

\usepackage[utf8]{inputenc}
\usepackage[english]{babel}
\usepackage[T1]{fontenc}

\usepackage{amsmath,amssymb,amsfonts}

\title{Computational Geometry - Exercises for week 2}
\author{Philip Munksgaard\\Sebastian Paaske Tørholm \\ Ejnar Håkonsen}

\begin{document}
\maketitle

\section{Exercise 3.8}

Adding diagonals to a DCEL can take $O(n)$ time since we have to
update the incident faces of all the edges in one half of the
diagonalized polygon, of which there may be up to $n-2$. Apart from
incident phases, adding a diagonal only requires us to modify two
\texttt{next} and \texttt{previous} pointers, as well as creating a
new face, all of which takes $O(1)$ time.

However, \texttt{MakeMonotone} changes the incident faces for each
edge at most once. Every time a diagonal is added, all the edges above
the diagonal are disregarded from the rest of the computations, and no
more diagonals are added to that part of the polygon. Since we know
that we add at most $n-2$ diagonals, the amortized cost of adding a
diagonal is thus $O(1)$.

\end{document}

