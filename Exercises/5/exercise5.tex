\documentclass[11pt,a4paper]{article}

\usepackage[utf8]{inputenc}
\usepackage[english]{babel}
\usepackage[T1]{fontenc}

\usepackage{amsmath,amssymb,amsfonts}
\usepackage[vlined, ruled, linesnumbered]{algorithm2e}

\title{Computational Geometry - Geometric Data Structures}
\author{Philip Munksgaard \\ Sebastian Paaske Tørholm \\ Ejnar Håkonsen}

\begin{document}
\maketitle

\section{Exercise 10.1}
\subsection{Pseudocode}
\begin{algorithm}[H]
	\caption{\FuncSty{BuildExerciseTree(}$lines$\FuncSty{)}}

	\SetKwFunction{BuildRangeTree}{Build1DRangeTree}
	\SetKwFunction{ConstructIntervalTree}{ConstructIntervalTree}
    \SetKwFunction{CanonicalSubset}{CanonicalSubset}
    \SetKwFunction{Assoc}{Assoc}
	\SetArgSty{}

    $ys \leftarrow $ y-coordiates of $lines$\;
    $T \leftarrow \BuildRangeTree(ys)$\;
    \ForEach{$node \in T$}{
        $ps \leftarrow $ points associated with $\CanonicalSubset(node)$\;
        $\Assoc(node) \leftarrow \ConstructIntervalTree(ps)$\;
    }

    \Return{$T$}\;
\end{algorithm}

\begin{algorithm}[H]
    \SetKwData{lines}{lines}

    \caption{\FuncSty{QueryExerciseTree(}$T,x,[y_1 : y_2]$\FuncSty{)}}

	\SetKwFunction{RangeQuery}{1DRangeQuery}
	\SetKwFunction{QueryIntervalTree}{QueryIntervalTree}
    \SetKwFunction{Assoc}{Assoc}
    \SetKwFunction{ReportAll}{ReportAll}
	\SetArgSty{}

    $nodes \leftarrow \RangeQuery(T, [y_1 : y_2])$\;
    \ForEach{$node \in nodes$}{
        $\ReportAll(\QueryIntervalTree(\Assoc(node), x))$\;
    }
\end{algorithm}

\subsection{Correctness}
% TODO

\subsection{Resource usage}
% TODO

\section{Exercise 10.6}

\subsection{Using segment trees}

We use a modified segment tree. At each node, however, instead of
saving the canonical subsets in each node, we save only the size of
the canonical subset. That way, we achieve $O(n)$ size, and we can
query the tree in the same manner as in \verb+QuerySegmentTree+, only
reporting the size of the canonical subset instead of the canonical
subset itself.

\section{Exercise 10.9}

\section{Exercise 10.10}

\end{document}

