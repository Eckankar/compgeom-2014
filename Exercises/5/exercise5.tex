\documentclass[11pt,a4paper]{article}

\usepackage[utf8]{inputenc}
\usepackage[english]{babel}
\usepackage[T1]{fontenc}

\usepackage{amsmath,amssymb,amsfonts}
\usepackage[vlined, ruled, linesnumbered]{algorithm2e}

\title{Computational Geometry - Geometric Data Structures}
\author{Philip Munksgaard \\ Sebastian Paaske Tørholm \\ Ejnar Håkonsen}

\begin{document}
\maketitle

\section{Exercise 10.1}
\subsection{Pseudocode}
\begin{algorithm}[H]
	\caption{\FuncSty{BuildExerciseTree(}$lines$\FuncSty{)}}

	\SetKwFunction{BuildRangeTree}{Build1DRangeTree}
	\SetKwFunction{ConstructIntervalTree}{ConstructIntervalTree}
    \SetKwFunction{CanonicalSubset}{CanonicalSubset}
    \SetKwFunction{Assoc}{Assoc}
	\SetArgSty{}

    $ys \leftarrow $ y-coordiates of $lines$\;
    $\mathcal{T} \leftarrow \BuildRangeTree(ys)$\; 
    \ForEach{$node \in \mathcal{T}$}{
        $ps \leftarrow $ points associated with $\CanonicalSubset(node)$\;
        $\Assoc(node) \leftarrow \ConstructIntervalTree(ps)$\;
    }

    \Return{$\mathcal{T}$}\;
\end{algorithm}

\begin{algorithm}[H]
    \SetKwData{lines}{lines}

    \caption{\FuncSty{QueryExerciseTree(}$\mathcal{T},x,[y_1 : y_2]$\FuncSty{)}}
	
	\SetKwFunction{RangeQuery}{1DRangeQuery}
	\SetKwFunction{QueryIntervalTree}{QueryIntervalTree}
    \SetKwFunction{Assoc}{Assoc}
    \SetKwFunction{ReportAll}{ReportAll}
	\SetArgSty{}

    $nodes \leftarrow \RangeQuery(\mathcal{T}, [y_1 : y_2])$\; 

    \ForEach{$node \in nodes$}{
        $\ReportAll(\QueryIntervalTree(\Assoc(node), x))$\;
    }
\end{algorithm}

\subsection{Correctness}
By correctness of range tree querying, any line that isn't returned by the query
to the range tree must have an y-coordinate outside of the interval $[y_1 : y_2]$.
Since such a line cannot cross our query line, such a line shouldn't be reported.

Furthermore, any line returned by the range query will have $y \in [y_1 : y_2]$,
and that each such line will be contained in precisely one of the reported subtrees.

By correctness of interval trees, we know that each interval belonging to the tree,
containing the queried $x$ coordinate, is reported precisely once. Since furthermore
each line corresponds to one interval, and is contained in precisely one interval tree
\footnote{By virtue of belonging to precisely one subtree in the range tree.}, we know
that each line segment with $y \in [y_1 : y_2]$, intersecting $x$, will be reported
exactly once, which was what was desired.

\subsection{Resource usage}
Since interval trees take up $O(n)$ space, we can use the same argument as is used
in lemma 5.6 to conclude that the range tree with associated interval trees takes up
$O(n \log n)$ space.

Constructing the outer range tree takes $O(n \log n)$. On each level, we construct a
number of interval trees with cost $O(k_i \log k_i)$, with $\sum k_i = n$. The sum of
the costs on each such level is bounded by $O(n \log n)$, and we have $\log n$ levels,
giving us a total construction cost of $O(n \log^2 n)$.

Querying the range tree costs $O(\log n)$, giving a list of length $O(\log n)$
subtrees. For each of these returned elements, we do a $O(\log n + k_i)$ query
against the associated interval tree, with $\sum k_i = k$ being the number of
found line segments. This gives a total of $O(\log^2 n + k)$.

\section{Exercise 10.6}

\subsection{Using segment trees}

We use a modified segment tree. At each node, however, instead of
saving the canonical subsets in each node, we save only the size of
the canonical subset. That way, we achieve $O(n)$ size, and we can
query the tree in the same manner as in \verb+QuerySegmentTree+, only
reporting the size of the canonical subsets instead of the canonical
subset itself.

\subsection{Using interval trees}

We use an unmodified interval tree to perform this query. To perform the
query, we look at the median position and compare it to our query point.
If the query point is on the left of the median, we use binary search on
$\mathcal{L}_left$ to to determine the first interval to contain the query
point. This gives us the number of intervals in $I_mid$ that cross the query
line. We then recurse on the left subtree, repeating the same approach.
If the point is on the right, the procedure is symmetrical.

$I_mid$ can at most contain all $n$ lines, so the binary search on this is
$O(\log n)$. Since our tree is divided on the median, $I_left$ and $I_right$
can contain at most $n$ points, and thus at most $n/2$ lines. We can therefore
express the running time for our query by the recurrence

\[ T(n) = T(n/2) + O(\log n), \]

which solves to $T(n) = O(\log n)$ by simple induction.

\subsection{Using binary search trees}

We use two binary search trees, one for the left edge of the
intervals, and one for the right edge of the intervals. In addition to
the coordinate, each node also stores the amount of left/right edges
left of it, such that each node in the tree for left edges stores the
amount of left edges to the left of that node, and each node in the
tree for right edges stores the amount of right edges to the left of
that it. Construction thus takes $O(n \log n)$ time, and uses $O(n)$
space.

To query, we first find the node corresponding to the highest valued
left edge and the node corresponding to the lowest valued right
edge. By simply subtracting the associated value for the right edge
from the associated value for the left edge, we get the amount of
crossing edges.

\section{Exercise 10.9}

We use a range tree with fractional cascading like the one from
exercise 5.10c. However, instead of $x$ and $y$ coordinates, we use
the left and right endpoints of the intervals, such that the range
tree indexes left endpoints, and the associated trees indexes the
right endpoints. To query for a given interval $[x:x']$, we first
search for nodes where the left endpoint is within the range and for
each such node we search in the associated tree for right edges that
are within the range. Then we simply check whether the interval
associated with each right edge found is in the query interval and
report the interval if that is the case.

Construction time is $O(n \log n)$, space is $O(n \log n)$ (like in
the 2 dimensional range trees) and query time is $O(\log n + k)$.

\section{Exercise 10.10}

We choose to use a priority search tree, in which we treat each
interval as a two-dimensional point, $(start, end)$.

Furthermore, we store the largest endpoint as $maxEnd$.

If we can perform a single query against this tree and get the desired
intervals, the storage and query constraints will be satisfied.

The query we desire is then simply for the interval \[
    (-\infty : x] \times [x' : maxEnd] 
\]

We simply desire any interval, where $start \leq x$ and
$end \geq x'$, which is precisely captured by this region.

\end{document}

