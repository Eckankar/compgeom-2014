\documentclass[11pt,a4paper]{article}

\usepackage[utf8]{inputenc}
\usepackage[english]{babel}
\usepackage[T1]{fontenc}

\usepackage{amsmath,amssymb,amsfonts}
\usepackage{hyperref}

\title{Computational Geometry - Delaunay Triangulations}
\author{Philip Munksgaard \\ Sebastian Paaske Tørholm \\ Ejnar Håkonsen}

\begin{document}
\maketitle

\section{Exercise 9.5}
\subsection{a}
We do this in two parts. First we show that the determinant being positive
reduces to a point being within a given circle. Second we show that this
circle is in fact the circle on which $p$, $q$ and $r$ lie.

\begin{align}
    \begin{vmatrix} p_x & p_y & p_x^2+p_y^2 & 1 \\
                    q_x & q_y & q_x^2+q_y^2 & 1 \\
                    r_x & r_y & r_x^2+r_y^2 & 1 \\
                    s_x & s_y & s_x^2+x_y^2 & 1 \end{vmatrix}
    &= - s_x \begin{vmatrix} p_y & p_x^2+p_y^2 & 1 \\
                             q_y & q_x^2+q_y^2 & 1 \\
                             r_y & r_x^2+r_y^2 & 1 \end{vmatrix}
       + s_y \begin{vmatrix} p_x & p_x^2+p_y^2 & 1 \\
                             q_x & q_x^2+q_y^2 & 1 \\
                             r_x & r_x^2+r_y^2 & 1 \end{vmatrix}\\
    & \quad\quad  - (s_x^2+s_y^2) \begin{vmatrix} p_x & p_y & 1 \\
                                       q_x & q_y & 1 \\
                                       r_x & r_y & 1 \end{vmatrix}
       + \begin{vmatrix} p_x & p_y & p_x^2+p_y^2 \\
                         q_x & q_y & q_x^2+q_y^2 \\
                         r_x & r_y & r_x^2+r_y^2 \end{vmatrix} \\
    &= - s_x a + s_y b - (s_x^2+s_y^2) c + d \\
    &= c (- s_x \frac{a}{c} + s_y \frac{b}{c} - (s_x^2+s_y^2) + \frac{d}{c})\\
    &= c (- (s_x + \frac{a}{2c})^2 - (s_y - \frac{b}{2c})^2 + \frac{4cd-a^2-b^2}{4c^2}) > 0 \\
    &\Leftrightarrow - (s_x + \frac{a}{2c})^2 - (s_y - \frac{b}{2c})^2 + \frac{4cd-a^2-b^2}{4c^2} > 0 \tag{Since c > 0.\footnote{\url{https://people.richland.edu/james/lecture/m116/matrices/applications.html}}} \\ 
    &\Leftrightarrow (s_x + \frac{a}{2c})^2 + (s_y - \frac{b}{2c})^2 < \frac{4cd-a^2-b^2}{4c^2}
\end{align}

This is true precisely if $s$ in contained within the circle
centred on $(-\frac{a}{2c}, \frac{b}{2c})$, with radius
$\sqrt{\frac{4cd-a^2-b^2}{4c^2}}$. We now need to verify that this is the
circle through $p$, $q$ and $r$.

From \url{http://everything2.com/title/A+circle+is+defined+by+three+points} we
get a formula for finding the centre and radius of the circle we desire. Using
Mathematica, we observe that the expressions simplify to be the same, and thus
the two formulations describe the same circle.

\subsection{b}


\section{Exercise 9.11}
\subsection{a}


\subsection{b}


\end{document}

