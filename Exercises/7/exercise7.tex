\documentclass[11pt,a4paper]{article}

\usepackage[utf8]{inputenc}
\usepackage[english]{babel}
\usepackage[T1]{fontenc}

\usepackage{amsmath,amssymb,amsfonts}
\usepackage{hyperref}

\title{Computational Geometry - Delaunay Triangulations}
\author{Philip Munksgaard \\ Sebastian Paaske Tørholm \\ Ejnar Håkonsen}

\begin{document}
\maketitle

\section{Exercise 9.5}
\subsection{a}
We do this in two parts. First we show that the determinant being positive
reduces to a point being within a given circle. Second we show that this
circle is in fact the circle on which $p$, $q$ and $r$ lie.

\begin{align}
    \begin{vmatrix} p_x & p_y & p_x^2+p_y^2 & 1 \\
                    q_x & q_y & q_x^2+q_y^2 & 1 \\
                    r_x & r_y & r_x^2+r_y^2 & 1 \\
                    s_x & s_y & s_x^2+x_y^2 & 1 \end{vmatrix}
    &= - s_x \begin{vmatrix} p_y & p_x^2+p_y^2 & 1 \\
                             q_y & q_x^2+q_y^2 & 1 \\
                             r_y & r_x^2+r_y^2 & 1 \end{vmatrix}
       + s_y \begin{vmatrix} p_x & p_x^2+p_y^2 & 1 \\
                             q_x & q_x^2+q_y^2 & 1 \\
                             r_x & r_x^2+r_y^2 & 1 \end{vmatrix}\\
    & \quad\quad  - (s_x^2+s_y^2) \begin{vmatrix} p_x & p_y & 1 \\
                                       q_x & q_y & 1 \\
                                       r_x & r_y & 1 \end{vmatrix}
       + \begin{vmatrix} p_x & p_y & p_x^2+p_y^2 \\
                         q_x & q_y & q_x^2+q_y^2 \\
                         r_x & r_y & r_x^2+r_y^2 \end{vmatrix} \\
    &= - s_x a + s_y b - (s_x^2+s_y^2) c + d \\
    &= c (- s_x \frac{a}{c} + s_y \frac{b}{c} - (s_x^2+s_y^2) + \frac{d}{c})\\
    &= c (- (s_x + \frac{a}{2c})^2 - (s_y - \frac{b}{2c})^2 + \frac{4cd-a^2-b^2}{4c^2}) > 0 \\
    &\Leftrightarrow - (s_x + \frac{a}{2c})^2 - (s_y - \frac{b}{2c})^2 + \frac{4cd-a^2-b^2}{4c^2} > 0 \tag{Since c > 0, see note 1.} \\ 
    &\Leftrightarrow (s_x + \frac{a}{2c})^2 + (s_y - \frac{b}{2c})^2 < \frac{4cd-a^2-b^2}{4c^2}
\end{align}

Note 1: From
\url{https://people.richland.edu/james/lecture/m116/matrices/applications.html
} we know that $c$ represents the area of the triangle $pqr$. Since the points
are given in clockwise order, we know $c$ to be positive.

We can now see, that the determinant being positive is equivanlent $s$ in
contained within the circle centred on $(-\frac{a}{2c}, \frac{b}{2c})$, with
radius $\sqrt{\frac{4cd-a^2-b^2}{4c^2}}$. We now need to verify that this is
the circle through $p$, $q$ and $r$.

From \url{http://everything2.com/title/A+circle+is+defined+by+three+points} we
get a formula for finding the centre and radius of the circle we desire. Using
Mathematica, we observe that the expressions simplify to be the same, and thus
the two formulations describe the same circle.

\subsection{b}
As per the definition of illegal edges, one could calculate the angles of
two possible triangulations of the, 4 points and see which set of angles was
minimal.

This would probably execute faster than computing the determinant of a 4 by
4 matrix, however it seems more prone to problems arising from numerical
precision.

\section{Exercise 9.11}
\subsection{a}

Suppose that there are two points, $p_1$ and $p_2$, that are not
connected in the Delaunay triangulation. By theorem 9.6(ii) we know
that two points form an edge in the Delaunay graph iff there is a
closed disc $C$ that contains the two points on its boundary, and does
not contain any other point of $P$. The minimal such circle must be
the one with $\overline{p_1p_2}$ as the diameter. If
$\overline{p_1p_2}$ is not connected in the Delaunay triangulation,
there must exist another point $p_3$, which is connected to $p_1$ and
$p_2$ inside $C$. That also means that $\overline{p_1p_3}$ and
$\overline{p_2p_3}$ will be shorter than $\overline{p_1p_2}$. But,
$p_1p_2p_3$ form a cycle in our graph, and from the cycle property of
EMSTs we know that the longest edge in a cycle cannot be a part of the
EMST. Thus, if $\overline{p_1p_2}$ are not in the Delaunay
triangulation for a given graph $P$, they are not in the EMST for $P$
either.

\subsection{b}

First we obtain a Delaunay triangulation of $P$ in $O(n \lg n)$ time. 

By lemma 9.11 we know such a triangulation to have $O(n)$ triangles.
We can consider triangulation to make up a graph with $O(n)$ edges.
Since we know an EMST of $P$ to be a spanning tree of the triangulation,
we must have that it also is a minimum spanning tree. 
Therefore, we can run Kruskal's algorithm in $O(n \lg n)$ on this graph,
and obtain a minimum spanning tree on the triangulation, and thus get
the desires EMST in $O(n \lg n)$.

\end{document}

