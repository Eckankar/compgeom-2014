\documentclass[11pt,a4paper]{article}

\usepackage[utf8]{inputenc}
\usepackage[english]{babel}
\usepackage[T1]{fontenc}

\usepackage{amsmath,amssymb,amsfonts}

\title{Computational Geometry - Point Location}
\author{Philip Munksgaard \\ Sebastian Paaske Tørholm \\ Ejnar Håkonsen}

\begin{document}
\maketitle

\section{Exercise 6.5}

Given an convex polygon $\mathcal{P}$ as an array of its $n$ vertices
in sorted order around the boundary, we wish to decide in $O(\log n)$
time whether a query point $q$ is inside the polygon.

We do this by a recursive binary search. Find the point in the middle
of the array, and determine whether or not the angle $\angle p_1
p_{n/2} q$ is a left turn or a right turn. If it is a left turn,
recursively go into the left half of the array (still using $p_1$ as
the origin point) and similarly for right turns. If at some point we
only have three points (and thus a triangle), checking whether $q$ is
inside that triangle is a constant time operation. Since we always
half the amount of vertices in our array, we are guaranteed to reach
the base case at some point, and we only use $O(\log n)$ time.

\section{Exercise 6.6}
Given a y-monotone polygon $\mathcal{P}$ as an array of its $n$ vertices
in sorted order around the boundary, we wish to decide in $O(\log n)$
time whether a query point $q$ is inside the polygon.

First we find the topmost and bottommost points $p_t$ and $p_b$ using binary
search on the polygon. On both sides, we now use binary search to locate the
points with y-coordinate immediately above ($p_{lu}$ and $p_{ru}$) and below
($p_{lb}$ and $p_{rb}$) $q$.

If $q$ is in $\mathcal{P}$, $q$ must be contained within the convex polygon
$(p_{lu}, p_{ru}, p_{rb}, p_{lb})$. Checking if the point is contained
within this can be done in constant time.

\section{Exercise 6.7}

\section{Exercise 6.16}

\end{document}

