\documentclass[11pt,a4paper]{article}

\usepackage[utf8]{inputenc}
\usepackage[english]{babel}
\usepackage[T1]{fontenc}

\usepackage{amsmath,amssymb,amsfonts}

\title{Computational Geometry - Exercises for week 1}
\author{Philip Munksgaard\\Sebastian Paaske Tørholm}

\begin{document}
\maketitle

\section{Exercise 1.3}
Let $E$ be the set of edges, and $\mathcal{P}$ be the (initially empty) list of
ordered points in the polygon.

First, create $P$, a sorted array of the endpoints of the edges, each tagged
with the edge they belong to. This takes $O(n \lg n)$.

Pick an edge $e_0$, to be our starting edge, and let $e_0 = (p_0, p_0')$.

Do a binary search on $P$ for $p_0$, finding the other edge ending in $p_0$. Let
$p_1$ be the other endpoint of this edge.

Check if $p_0', p_0$ and $p_1$ together make a right turn. If so, add $p_0'$ and
$p_0$ to $\mathcal{P}$. Let $p_{start} = p_0'$.

If they do not make a right turn, locate the edge corresponding to $p_0'$ in
$P$, and let $p_1'$ be the other edgepoint, in a similar fashion as to what
was done for $p_1$. $p_0$, $p_0'$ and $p_1'$ must now make a right turn, so add
$p_0$ and $p_0'$ to $\mathcal{P}$. Let $p_{start} = p_0$.

For $i$ starting at $1$, ending when $p_i \neq p_{start}$, repeat the following: 
\begin{itemize}
    \item Add $p_i$ to $\mathcal{P}$
    \item Find the other edge containing $p_i$ using binary search on $P$. Let $p_{i+1}$
          be the other endpoint of this line.
\end{itemize}

After this, $\mathcal{P}$ contains the clockwise polygon corresponding to the
edges in $E$.

\section{Exercise 1.6}

\subsection{a}
Let $L$ be a set of line segments and $E$ be the set of endpoints of the lines in $L$.

We wish to prove that $CH(L) = CH(E)$.

Since $E \subseteq L$, and $L \subseteq CH(L)$, we must have $E \subseteq CH(L)$. Since
$CH(E)$ is the smallest such convex set, we must have that $CH(E) \subseteq CH(L)$.

We now need that $CH(L) \subseteq CH(E)$. By definition of convexity, we know that for any
$p_1$ and $p_2$ in $E$ the line $p_1p_2 \subseteq CH(E)$. Since each line in $L$ has this
form, we must have that $L \subseteq CH(E)$. By same argument as above, we get $CH(L) \subseteq CH(E)$.

This proves that $CH(L) = CH(E)$.

\subsection{b}
% TODO

\section{Chan's paper}
% TODO


\end{document}

