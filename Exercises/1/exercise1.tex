\documentclass[11pt,a4paper]{article}

\usepackage[utf8]{inputenc}
\usepackage[english]{babel}
\usepackage[T1]{fontenc}

\usepackage{amsmath,amssymb,amsfonts}

\title{Computational Geometry - Exercises for week 1}
\author{Philip Munksgaard\\Sebastian Paaske Tørholm}

\begin{document}
\maketitle

\section{Exercise 1.3}
Let $E$ be the set of edges, and $\mathcal{P}$ be the (initially empty) list of
ordered points in the polygon.

First, create $P$, a sorted array of the endpoints of the edges, each tagged
with the edge they belong to. This takes $O(n \lg n)$.

Pick an edge $e_0$, to be our starting edge, and let $e_0 = (p_0, p_0')$.

Do a binary search on $P$ for $p_0$, finding the other edge ending in $p_0$. Let
$p_1$ be the other endpoint of this edge.

Check if $p_0', p_0$ and $p_1$ together make a right turn. If so, add $p_0'$ and
$p_0$ to $\mathcal{P}$. Let $p_{start} = p_0'$.

If they do not make a right turn, locate the edge corresponding to $p_0'$ in
$P$, and let $p_1'$ be the other edgepoint, in a similar fashion as to what
was done for $p_1$. $p_0$, $p_0'$ and $p_1'$ must now make a right turn, so add
$p_0$ and $p_0'$ to $\mathcal{P}$. Let $p_{start} = p_0$.

For $i$ starting at $1$, ending when $p_i \neq p_{start}$, repeat the following: 
\begin{itemize}
    \item Add $p_i$ to $\mathcal{P}$
    \item Find the other edge containing $p_i$ using binary search on $P$. Let $p_{i+1}$
          be the other endpoint of this line.
\end{itemize}

After this, $\mathcal{P}$ contains the clockwise polygon corresponding to the
edges in $E$.

\section{Exercise 1.6}

\subsection{a}
Let $L$ be a set of line segments and $E$ be the set of endpoints of the lines in $L$.

We wish to prove that $CH(L) = CH(E)$.

Since $E \subseteq L$, and $L \subseteq CH(L)$, we must have $E \subseteq CH(L)$. Since
$CH(E)$ is the smallest such convex set, we must have that $CH(E) \subseteq CH(L)$.

We now need that $CH(L) \subseteq CH(E)$. By definition of convexity, we know that for any
$p_1$ and $p_2$ in $E$ the line $p_1p_2 \subseteq CH(E)$. Since each line in $L$ has this
form, we must have that $L \subseteq CH(E)$. By same argument as above, we get $CH(L) \subseteq CH(E)$.

This proves that $CH(L) = CH(E)$.

\subsection{b}
% TODO

\section{Chan's paper}

Chan's paper describes an algorithm for finding the convex hull of of
a set of objects in two and three dimensions using $O(n log h)$ time and
linear space. Here, we will focus on the two-dimensional case.

Several other algorithms for computing convex hulls in $O(n log h)$ time
have been presented, but they can be quite complicated to implement,
and some of them hide big constants under the hood. Chan's proposed
algorithm, on the other hand, uses simple techniques while achieving
the same worst-case time complexity.

The algorithm is simple in its description: split the set of points
into groups of size $m$, where $0 < m \leq m$; compute the convex hull
of each group using another simple CV algorithm, such as Graham's
scan; and merge the groups using a gift-wrapping algorithm. Computing
the convex hull for each group using Graham's scan takes $O(m log m)$, so
the trick lies in picking an appropriate value for $m$ and performing
the merge in an effective way.

Let us first consider the merging operation. We are given $\lceil n /
m \rceil$ convex sets of at most $m$ points, and we wish to compute
the convex hull of all the points. Given an initial point, $p_1$,
known to be in the convex hull (such as the rightmost point of them
all), and a point infinitely far away, $p_0$, for each group $i$, find
the point, $q_i$ such that the angle $\angle p_0 p_1 q_i$ is
maximized. Since the groups are given as a list of points in
counter-clockwise order, we can find $q_i$ by using a binary
search. If we ignore the details of the binary search for the moment,
and concentrate on the fact that it will give us a $log m$ search time
for each $q_i$, we then have to find the $q$ out of all the $g_i$s,
that maximizes the angle, which takes $O(n / m)$ time. Finally, we use
this $q$ as our next initial point and $p_1$ as the previous point and
continue the search. We perform this search a maximum of $H$ times,
for a total of $O(H((n / m) log m))$ time. Thus, by choosing $H=m$,
merging takes $O(m log m)$ time.

The binary search is not described in Chan's paper, only referencing
another paper~\footnote{\emph{An algorithm for convex polytopes} by
  D.R. Chand and S.S. Kapur}, which does not explain the method very
well either. The basic idea however, is to take advantage of the fact
that the coordingates are given in clockwise or counter-clockwise
order. Then, we continually check whether the tangent point is in the
left or right half of some intersection of the polygon. For a more
in-depth explanation, including some example code which actually
performs the binary search, consult Tom Switzer's blog post on the
matter~\footnote{\url{http://tomswitzer.net/2010/12/2d-convex-hulls-chans-algorithm/},
  accessed 2014-02-10}.

Finally, we just have to chose an appropriate value for $H$. By simply
trying successive values for $2^2^t$ until the above algorithm does
not return \emph{incomplete}, we use at most $log log h$ tries to find
the appropriate value for $H$, and with each iteration taking $O(n log
h)$ time, the overall time complexity is $O(n log h)$.

\end{document}

